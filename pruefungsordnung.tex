\documentclass[a4paper,10pt]{article}
\usepackage[utf8x]{inputenc}
\usepackage{geometry}
\usepackage{tabularx}
\usepackage{graphicx}
\usepackage[hang, splitrule]{footmisc}
\usepackage{wallpaper}
\usepackage{wrapfig}
\usepackage[ngerman]{babel}

\CenterWallPaper{0.8}{./inphima_final_grau}




\setlength{\parindent}{0mm}
\setlength{\parskip}{2mm}
\addtolength{\footskip}{0.5cm}
\setlength{\footnotemargin}{0.3cm}
\setlength{\footnotesep}{0.4cm}

\pagestyle{empty}

\title{Prüfungsordnung der ESAG-Rallye}
\author{Fachschaften Informatik, Mathematik, (medizinische) Physik}
\date{06.10.2011}


\begin{document}
\pagestyle{plain}
\maketitle
\tableofcontents

\newpage
\pagestyle{plain}
\section[Formalitäten]{Formalitäten zuerst}

\subsection{Umfang der Rallye und Punktevergabe}

Die ESAG-Rallye umfasst in diesem Jahr zehn Prüfungen. Alle Teilnehmerinnen 
und Teilnehmer werden vorab in Gruppen eingeteilt, die bezüglich Alter, Geschlecht und
Studienfach so ausgewogen wie möglich von den drei veranstaltenden
Fachschaftsräten gebildet werden. Die Anzahl der Teilnehmerinnen und Teilnehmer pro Gruppe
wird anhand der Anzahl aller Teilnehmerinnen und Teilnehmer spontan bestimmt.

\begin{wrapfigure}{l}{0.19\textwidth}
\begin{tabular}{c|c}
\textbf{Pos.} & \textbf{Pkt.} \\
\hline
1 & 25\\
2 & 21 \\
3 & 19 \\
4 & 17 \\
5 & 15 \\
6 & 13 \\
7 & 11 \\
8 & 10 \\
9 & 9 \\
10 &8 \\
11 & 7 \\
12 & 6 \\
13 & 5 \\
14 & 4 \\
15 & 3\\
16 & 2\\
17 & 1\\
18 & 0 
\end{tabular}
\vspace{15pt}
\end{wrapfigure}

Den Teilnehmerinnen und Teilnehmern wird für jede Station ein Zeitraum 
von 20 Minuten eingeräumt. Innerhalb dieses Zeitraums muss auch der 
Raumwechsel sowie die Erklärung der Spielregeln erfolgen. Zwischen 
13 und 14 Uhr ist eine Pause vorgesehen. Der genaue Zeitpunkt der Pause 
kann dem invidiuellen Laufzettel entnommen werden. Die Prüfungen müssen 
zu den auf dem Laufzettel vorgegebenen Zeitpunkten angetreten werden.


Die Punktevergabe erfolgt erst im Anschluss an die Rallye. Für jede
Prüfung wird nach geeigneten Kriterien eine Rangliste aller
Teilnehmergruppen ermittelt, die natürlich erst dann vollständig ist,
wenn alle Gruppen die Prüfung abgelegt haben oder die Rallye allgemein
für beendet erklärt wird. Diese Ranglisten bestimmen die Punkte, die
eine Gruppe für eine Prüfung erhält.

Nebenstehende Tabelle liegt der Punktevergabe zugrunde. Bei $x$ gleich
guten Gruppen erhalten diese die höhere Punktzahl und es werden $x-1$ Plätze
übersprungen.

Den verantwortlichen Prüferinnen und Prüfern ist es erlaubt, besondere
Leistungen einzelner Gruppen oder einzelner Gruppenmitglieder durch die
Vergabe von Bonuspunkten zu belohnen.

Die Gruppe, die in der Summe über alle Prüfungen die meisten Punkte
erhalten hat, gewinnt die Rallye. Bei Punktgleichheit entscheidet die
Anzahl der Erstplatzierungen bei den Prüfungen. Steht die Siegergruppe
dann noch nicht fest, entscheidet eine Schätzfrage. Die Schätzungen
werden ohne Kenntnis der jeweils anderen Schätzungen abgegeben.

\subsection{Teilnahmebedingungen}

Teilnahmeberechtigt sind alle Studierenden, die zum Wintersemester 
2011/2012 zum ersten Mal ein Studium der
Informatik, der Mathematik, der Physik oder der medizinischen Physik an
der Heinrich-Heine-Universität in Düsseldorf beginnen und zum Zeitpunkt
der Rallye das 16. Lebensjahr vollendet haben.

Vor den Folgen erhöhten Alkoholkonsums wird an dieser Stelle
ausdrücklich gewarnt. Wer entgegen ausdrücklicher Vorwarnungen weder zu
Fuß noch mit dem ÖPNV angereist ist, sollte auf Alkoholkonsum während
der Rallye oder auf eine Rückreise auf dem selben Weg am selben Tag
verzichten. Die Teilnahme an der Rallye ohne ein Handtuch ist grob
fahrlässig.

Trotz größtmöglicher Sorgfalt in der Vorbereitung und Durchführung der
Rallye sind die Teilnehmerinnen und Teilnehmer für ihr Wohlergehen 
jederzeit selbst verantwortlich. Unterstützt werden die Teilnehmerinnen
und Teilnehmer durch die Möglichkeit des Konsums diverser Nahrungsmittel. 
Im Anschluss an die Rallye steht frisch Gegrilltes zur Verfügung.

Art und Umfang der Rallye können von den veranstaltenden Fachschaftsräten
zu jeder Zeit geändert werden. Ein Anspruch auf Teilnahme seitens der 
Teilnehmerinnen und Teilnehmer besteht nicht.

\subsection{Prüfungszulassung}

Für die Zulassung zu den einzelnen Teilprüfungen ist ein Laufzettel erfoderlich, 
der zu Beginn der Rallye an jede Gruppe ausgeteilt wird.

\subsection{Beteiligungsnachweise}


Die erzielten Prüfungsleistungen werden auf elektronischem Weg an das
Prüfungsamt übermittelt. Zusätzlich ist die erreichte Punktzahl auf dem 
Laufzettel der Gruppe einzutragen und von der jeweiligen Prüferin oder dem
jeweiligen Prüfer abzustempeln. 
Der Laufzettel wird nach der neunten Teilprüfung an die Prüferin oder den Prüfer 
der aktuellen Teilprüfung gegeben.

Gefälschte Prüfungsnachweise können zur Disqualifikation führen.

\subsection{Regelverletzungen}

Regeln gelten als solche, sobald sie durch ein Mitglied der drei 
veranstaltenden Fachschaftsräte bekannt gemacht werden. Inhalte dieser
Prüfungsordnung gelten als Regeln. Eine Verletzung der Regeln kann zur
Disqualifikation führen. Wer sich unfair gegenüber anderen Teilnehmern
verhält, verstößt gegen die Regeln. Eine ungültige Regel beeinträchtigt
nicht die Gültigkeit anderer Regeln.



\section{Beginn der Rallye}

Die ESAG-Rallye 2011 beginnt spätestens jetzt. Sollte dieser Prüfungsordnung
kein Laufzettel beiliegen, haben sich die Teilnehmerinnen und Teilnehmer an 
einen der Verantwortlichen. Die Teilnehmerinnen und Teilnehmer begeben sich 
sofort und nicht erst zum angegebenen Zeitpunkt zur ersten Teilprüfung.

\section{Ablauf der Rallye}
Nach Beginn der Rallye und dem Besuch der ersten Teilprüfung werden alle 
weiteren Teilprüfungen zu den auf dem Laufzettel vorgegebenen Zeitpunkten 
angetreten. Verspätet sich eine Teilnehmergruppe sind die jeweiligen Prüferinnen
und Prüfer berechtigt, die Gruppe von der Teilprüfung auszuschließen. Dies 
gilt insbesondere dann, wenn durch die Verzögerung weitere Verzögerungen im 
Ablauf zu erwarten sind. Im Anschluss an die neunte Teilprüfung werden die 
Teilnehmerinnen und Teilnehmer zur abschließenden Prüfung und zur Bekanntgabe 
der Ergebnisse in einen Hörsaal gebracht. Die Teilnahme am nachfolgenden 
Grillen ist nicht obligatorisch, wird aber durch das Prüfungsamt empfohlen.

\section[Informationen]{Wichtige Informationen}
\subsection{Kennzeichnung der Prüfungsorte }
Alle Prüfungsorte sind durch Luftballons an ihren Eingängen kenntlich gemacht. 
Zu einigen Prüfungen existieren mehrere Prüfungsorte. Welchen Prüfungsort die 
Teilnehmerinnen und Teilnehmer aufsuchen müssen, entnehmen sie ihren Laufzetteln.

\subsection{Veranstaltende Fachschaften}

\textbf{Fachschaft Informatik}\\
Raum 25.12.O1.18\\
Tel.: 0211 81-14846\\
fscs@uni-duesseldorf.de\\
www.hhu-fscs.de

\textbf{Fachschaft Mathematik}\\
Raum 25.22.U1.25\\
Tel.: 0211 81-13607\\
kontakt@fsmathe.de\\
www.fsmathe.de

\textbf{Fachschaft Physik und medizinische Physik}\\
Raum 25.32.O0.21\\
Tel.: 0211 81-13232\\
fsphysik@uni-duesseldorf.de\\
www.fsphy.uni-duesseldorf.de




\end{document}
