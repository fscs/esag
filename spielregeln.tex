\documentclass[a4paper,10pt]{article}
\usepackage[utf8x]{inputenc}
\usepackage{geometry}
\usepackage{tabularx}
\usepackage{graphicx}
\usepackage[ngerman]{babel}
\usepackage{wallpaper}

\CenterWallPaper{0.8}{./inphima_final_grau}

\pagestyle{empty}

\def\spiela{Brücken bauen}
\def\rauma{25.22.U1.33}
\def\spielb{Extremsport}
\def\raumb{DRAUSSEN}
\def\spielc{Hochstapeln}
\def\raumc{25.22.O1.18}
\def\spield{In$\Phi $Ma-Duell} 
\def\raumd{5L}
\def\spiele{Inverse}
\def\raume{25.13.U1.22}
\def\raumee{25.13.U1.24}
\def\spielf{Jeopardy}
\def\raumf{5K}
\def\spielg{Memory}
\def\raumg{25.32.O0.21}
\def\raumgg{25.22.U1.25}
\def\spielh{Parabelflug}
\def\raumh{DRAUSSEN}
\def\spieli{Pi$\times $Daumen}
\def\raumi{25.22.U1.34}
\def\spielj{Kleiderkette}
\def\raumj{}

\geometry{a4paper, top=20mm, left=15mm, right=15mm, bottom=20mm}
\renewcommand*\arraystretch{1.5}



\newcommand{\unten}{
 Bitte notiert die Punktzahl der Gruppe sowohl auf eurem Übersichtszettel als auch auf dem Zettel der Gruppe und stempelt letzteren ab. Wenn möglich tragt die Punktzahl außerdem bitte in der Weboberfläche ein.
 Achtet darauf, dass die vorgegebene Zeit nicht überschritten wird, da die Gruppen sonst nicht pünktlich zu ihrer nächsten Station kommen. Ihr dürft auf keinen Fall noch mit einer Gruppe spielen, wenn die nächste laut Plan bereits an der Reihe ist.
}

\parindent 0pt
\begin{document}

%%%Brücken bauen%%%

  \begin{tabularx}{\textwidth}{lc}
    \includegraphics[width=0.2\textwidth]{hhulogo_g}
  & {\Huge \textbf{Spielregeln: \spiela}}
  \end{tabularx}\\

\Large
\section*{Für die Spieler:} 
Ihr habt zehn Minuten Zeit, um mit einem Kilogramm Zeitung eine Brücke zu bauen, mit der ihr die Lücke zwischen diesen beiden Tischen überbrückt. Die Lücke ist etwa 45 cm groß. Ihr dürft ausschließlich Zeitungen zum Bau der Brücke verwenden. Die Brücke muss auf beiden Tischen aufliegen. Nach zehn Minuten wird ein Eimer auf die Brücke gestellt und so lange Gewicht in den Eimer gelegt, bis die Brücke einstürzt. Das Gewicht, welches die Brücke zuletzt getragen hat, ergibt eure Punktzahl bei diesem Spiel. 

\section*{Für die Spielleiter:}
\subsection*{Requisiten:}
\begin{itemize}
 \item Eimer
 \item 10 Pakete Zucker
 \item min. 18kg Zeitungen
\end{itemize}
\subsection*{Punkte:}
Anzahl der Zuckertüten, die in den Eimer gelegt werden konnten, bevor die Brücke einbrach. 10, falls alle Tüten getragen wurden.

\subsection*{Hinweise:}  
\unten

\newpage

%%%Extremsport

  \begin{tabularx}{\textwidth}{lc}
    \includegraphics[width=0.2\textwidth]{hhulogo_g}
  & {\Huge \textbf{Spielregeln: \spielb}}
  \end{tabularx}\\

\Large
\section*{Für die Spieler:}
Ihr müsst einen aus vier Etappen bestehenden Staffellauf bewältigen. 
Der Staffellauf besteht aus Eierlauf, Sackhüpfen, einem Dosenparcours und 
Dreibeinlauf. Sobald eine Etappe bewältigt wurde, darf die nächste begonnen
werden. Für umgeworfene Hindernisse werden Strafsekunden zu eurer Zeit 
addiert. Zwei Teams starten gleichzeitig an gegenüberliegenden Etappen.

\section*{Für die Spielleiter:}
\subsection*{Punkte:}
Gemessene Zeit inklusive Strafsekunden. Je kürzer desto besser.

\subsection*{Hinweise:}
\unten

\newpage

%%%Hochstapeln

  \begin{tabularx}{\textwidth}{lc}
    \includegraphics[width=0.2\textwidth]{hhulogo_g}
  & {\Huge \textbf{Spielregeln: \spielc}}
  \end{tabularx}\\

\Large
\section*{Für die Spieler:}
Ihr habt 10 Minuten Zeit. Innerhalb dieser 10 Minuten müssen Bausteine unten aus dem Turm genommen werden und oben wieder aufgelegt werden. Steine dürfen aus jeder Ebene unterhalb der obersten vollständigen Ebene entnommen werden. Zum Spielen darf nur eine Hand verwendet werden. Ihr spielt reihum, es kommt also jeder dran. Solange der Turm sicher steht, wird gezählt, wie viele Steine ihr aufstapelt. Fällt der Turm um, wird eure bisherige Punktzahl für diesen Turm quadriert. Wer zuletzt den Turm berührt hat, muss trinken. Ihr dürft den Turm nun neu aufbauen und von vorne anfangen, bis die Zeit abgelaufen ist. Punkte, die ihr einmal erreicht habt verliert ihr nicht mehr.

\section*{Für die Spielleiter:}
\subsection*{Requisiten:}
\begin{itemize}
 \item 2 Jenga Türme
 \item Stoppuhr
\end{itemize}
\subsection*{Punkte:}
Sei der Turm $n$ mal wieder aufgebaut worden und $t_i$ die Anzahl Steine, die in der $i$-ten Runde auf den Turm gelegt worden, bevor dieser umkippte.
 $$\sum_{i=1}^{n}t_i^2$$

\subsection*{Hinweise:}
\unten

\newpage

%%%InPhiMa-Duell

  \begin{tabularx}{\textwidth}{lc}
    \includegraphics[width=0.2\textwidth]{hhulogo_g}
  & {\Huge \textbf{Spielregeln: \spield}}
  \end{tabularx}\\

\Large
\section*{Für die Spieler:}
Im In$\Phi$Ma-Duell treten zwei Gruppen gleichzeitig an. Wie in der bekannten TV-Show Familienduell haben auch wir fast 100 Leute zu verschiedenen Themen gefragt und ihre Antworten gesammelt. Die Fragen, die wir Ihnen gestellt haben nennen wir euch ebenfalls. Jede Gruppe schreibt die Antwort, von der sie glaubt, dass sie am häufigsten genannt wurde, auf ihre Tafel. Ihr bekommt Punkte, wenn diese Antwort unter den Top Antworten war, und zwar so viele, wie Menschen sie genannt haben. Anschließend stellen wir euch die Frage ein weiteres Mal und ihr dürft eine weitere (noch nicht genannte) Antwort geben. Für jede Antwort habt ihr 20 Sekunden Zeit. Wir spielen maximal 10 Minuten oder 15 Fragen lang.

\section*{Für die Spielleiter:}
\subsection*{Requisiten:} 
\begin{itemize}
 \item Laptop
 \item Beamer
 \item Lautsprecher
 \item Tafeln
 \item Kreide
\end{itemize}

\subsection*{Punkte:}
Für die Anzahl an Menschen die die selbe Antwort gegeben haben wie die Gruppe. Es ist nicht nötig, dass der selbe Wortlaut gewählt wurde! Die Punkte aus allen Runden werden addiert. 

\subsection*{Hinweise:}
\unten

\newpage

%%%Inverses

  \begin{tabularx}{\textwidth}{lc}
    \includegraphics[width=0.2\textwidth]{hhulogo_g}
  & {\Huge \textbf{Spielregeln: \spiele}}
  \end{tabularx}\\

\Large
\section*{Für die Spieler:}
Es gibt zwei Spielleiter und drei Aktionen mit zugehörigere gegenteiliger Aktion: 
\begin{itemize}
 \item Hut an- oder ausziehen
 \item trinken oder nicht trinken
 \item sitzen oder stehen
\end{itemize}
Einer der Spielleiter macht euch eine beliebige Abfolge dieser Aktionen vor und ihr müsst immer das Gegenteil tun. Der andere Spielleiter stoppt dies nach einer pro Runde länger werdenden Zeit. Jeder muss zu diesem Zeitpunkt erstarren. Alle, die nicht genau das Gegenteil von dem tun, was der Spielleiter gerade tut scheiden aus. Eure Punkte bekommt ihr nach folgender Formel:\\
$$\sum\limits_{i=1}^{10}i \cdot |\mbox{Mitspieler in Runde}  (i+1)|$$


\section*{Für die Spielleiter:}
\subsection*{Requisiten:}\\
\begin{itemize}
 \item Becher und Getränke
 \item Zeitungshüte
 \item Stühle 
\end{itemize}
\subsection*{Punkte:}
$$\sum\limits_{i=1}^{10}i \cdot |\mbox{Mitspieler in Runde} (i+1)|$$

\subsection*{Hinweise:}
\unten

\newpage

%%%Jeopardy

  \begin{tabularx}{\textwidth}{lc}
    \includegraphics[width=0.2\textwidth]{hhulogo_g}
  & {\Huge \textbf{Spielregeln: \spielf}}
  \end{tabularx}\\

\Large
\section*{Für die Spieler:}
Es spielen zwei Gruppen gegeneinander. Jede Gruppe wählt abwechselnd eine Kategorie und Schwierigkeitsstufe. Daraufhin wird euch allen eine Antwort gezeigt und ihr (beide Gruppen) müsst die zugehörige Frage auf eure Tafel schreiben. Für jede Frage habt ihr Zeit, bis die Jeopardy Melodie abgelaufen ist. Ihr spielt für 10 Minuten und beantwortet maximal 10 Antworten. Ist die gewählte Frage korrekt bekommt ihr die zugehörige Punktzahl guteschrieben sonst wird sie abgezogen. Ihr könnt nie unter 0 Punkte fallen. Antwortet eine Gruppe falsch muss ein Gruppenmitglied einen Becher Bier/Wasser trinken, dieses Mitglied darf sich nicht mit den anderen beraten, bis der Becher leer ist, alle anderen spielen weiter.

\section*{Für die Spielleiter:}
\subsection*{Requisiten:} 
\begin{itemize}
 \item Laptop und Beamer
 \item Lautsprecher
 \item Tafeln
 \item Kreide
 \item Getränke und Becher
\end{itemize}

\subsection*{Punkte:}
Aufaddieren, wenn Frage gewählten Schwierigkeitsgrads richtig war, sonst abziehen.
\subsection*{Hinweise:}
\unten

\newpage

%%%Memory

  \begin{tabularx}{\textwidth}{lc}
    \includegraphics[width=0.2\textwidth]{hhulogo_g}
  & {\Huge \textbf{Spielregeln: \spielg}}
  \end{tabularx}\\

\Large
\section*{Für die Spieler:}
Ihr spielt Memory. Innerhakb von 10 Minuten versucht ihr so viele Paare wie möglich zu finden. Die ersten 4 Kartenpaare die ihr aufdeckt sind frei. Danach muss einer aus der Gruppe für jedes falsch aufgedeckte Kartenpaar ein Glas Wasser/Bier trinken. Die Gruppe muss mit dem Weiterspielen warten, bis dieser fertig ist.

\section*{Für die Spielleiter:}
\subsection*{Requisiten:} 
\begin{itemize}
 \item Laptop
 \item Beamer
 \item Getränke
 \item Becher
\end{itemize}
\subsection*{Punkte:}
Anzahl der korrekt aufgedeckten Paare.
\subsection*{Hinweise:}
Versucht immer schon genug Getränke eingeschenkt zu haben, um in der Zeit zu bleiben.\\
\unten

\newpage

%%%Parabelflug

  \begin{tabularx}{\textwidth}{lc}
    \includegraphics[width=0.2\textwidth]{hhulogo_g}
  & {\Huge \textbf{Spielregeln: \spielh}}
  \end{tabularx}\\


\Large
\section*{Für die Spieler:}
Ihr habt so viele Würfe, wie die größte Gruppe Teilnehmer. Jedes Gruppenmitglied wirft mindestens einmal auf das Zielfeld und eine Dosenpyramide (aus drei Dosen). Es dürfen so viele Personen mehrfach werfen, bis n-mal geworfen wurde. Jede komplett umgeworfene Dosenpyramide erhöht den Multiplikator x (zu Beginn x=1) um eins.  Die im Zielwerfen erlangten Punkte werden mit x multipliziert und ergeben die Gesamtpunktzahl.


\section*{Für die Spielleiter:}
\subsection*{Requisiten:} 
\begin{itemize}
 \item Kreppband
 \item Tennisbälle
 \item Reissäckchen
 \item Dose
 \item Besen
\end{itemize}

\subsection*{Punkte:}
Sei x die Anzahl der komplett umgeworfenden Dosenpyramiden und y die Punkte, die im Zielwerfen werreicht werden. Die Punktzahl ist dann $(x+1)*y$. 
\subsection*{Hinweise:}
Bitte fegt Reis, der aus platzenden Säckchen fällt, auf.
\unten

\newpage

%%%PiMalDaumen
  \begin{tabularx}{\textwidth}{lc}
    \includegraphics[width=0.2\textwidth]{hhulogo_g}
  & {\Huge \textbf{Spielregeln: \spieli}}
  \end{tabularx}\\


\Large
\section*{Für die Spieler:}
Ihr bekommt Schätzfragen gestellt und müsst versuchen, diese so genau wie möglich
zu beantworten. 

\section*{Für die Spielleiter:}
\subsection*{Requisiten:} 
\subsection*{Punkte:}
\subsection*{Hinweise:}
\unten

\newpage
\end{document}
