\documentclass[a4paper,10pt]{article}
\usepackage[utf8x]{inputenc}
\usepackage{geometry}
\usepackage{tabularx}
\usepackage{graphicx}
\usepackage[ngerman]{babel}
\usepackage{wallpaper}

\CenterWallPaper{0.8}{./inphima_final_grau}

\pagestyle{empty}

\geometry{a4paper, top=20mm, left=15mm, right=15mm, bottom=20mm}
\renewcommand*\arraystretch{1.5}

\parindent 0pt
\begin{document}

\begin{center}
{\Huge \textbf{Fachschaft Mathe: Memory}}
\end{center}
\vspace{2cm}
{\LARGE
Ihr spielt Memory. Innerhalb von 7 Minuten versucht ihr so viele Paare wie möglich zu finden. Die ersten 4 Kartenpaare, die ihr aufdeckt, sind frei. Danach muss einer aus der Gruppe für jedes falsch aufgedeckte Kartenpaar ein Glas Wasser/Bier trinken. Die Gruppe muss mit dem Weiterspielen warten, bis dieser fertig ist.
Eure Punkte ergeben sich wie folgt: \vspace{1cm}
    \begin{center} 

    gefundene Paare $\cdot$ 100 + 99 - falsche Paare + 10 $\cdot$ verbleibende Sekunden 
     
    \end{center}

}

\pagebreak

\begin{center}
{\Huge \textbf{Fachschaft Info: Turmbau}}
\end{center}
\vspace{2cm}
{\LARGE
An dieser Station errichtet ihr einen Turm. Zur Errichtung des Turm habt ihr 5 Minuten Zeit, eure Punkte ergeben sich aus der Höhe des errichteten Turms. Der Turm muss nach Abschluss noch 10 Sekunden stehen bleiben. In der Fachschaft findet ihr hierzu folgende Materialien:
\begin{itemize}
 \item 2 Cubbs
 \item 2 Wurfhölzer
 \item 6 Spielkarten
 \item 1 Ordner
 \item 1 leere Flasche
 \item 1 Frisbeescheibe
\end{itemize}
Außerdem dürft ihr insgesamt (d.h. pro Gruppe) 4 eurer Kleidungsstücke benutzen.
}

\pagebreak

\begin{center}
{\Huge \textbf{Fachschaft Physik: Singstar}}
\end{center}
\vspace{2cm}
{\LARGE
In dieser Fachschaft wird Singstar gespielt. Es singen 4 eurer Gruppenmitglieder. Bitte sucht euch schon vorher einen Song aus (Liste siehe unten). Eure Punktzahl ist die erreichte Punktzahl im Spiel.
\vspace{1cm}
\begin{itemize}
 \item Cruel Angel's Thesis - Yoko Takahashi

 \item Dancing Queen - Abba

 \item Barbie Girl - Aqua

 \item Eye of the Tiger - At Vance

 \item Lemon Tree - Fool's Garden

 \item Walking on Sunshine - Katrina and the Waves

 \item Hardrock Hallelujah - Lordi

 \item Do Wah Diddy Diddy - Manfred Mann

 \item Cheri Cheri Lady - Modern Talking

 \item Always Look on the Bright Side of Life - Monty Python

 \item 99 Red Balloons - Nena

 \item Wishmaster - Nightwish

 \item Never Gonna Give You up - Rick Astley

 \item Y.M.C.A. - Village People

 \item Bumbibjörnarna (Titelsong Gummibärenbande) - Peter Jöback

\end{itemize}

}

\pagebreak

\begin{center}
 {\Huge \textbf{Punktetabelle Mathe}} \vspace{2cm} \\
 \LARGE
 \begin{tabular}{|r|l|l|}
  \hline
  Gruppe & Punkte & Platz  \\ \hline
  1 &\hspace{5cm} & \hspace{3cm}\\ \hline 
  2 &\hspace{5cm} & \hspace{3cm}\\ \hline
  3 &\hspace{5cm} & \hspace{3cm}\\ \hline
  4 &\hspace{5cm} & \hspace{3cm}\\ \hline
  5 &\hspace{5cm} & \hspace{3cm}\\ \hline
  6 &\hspace{5cm} & \hspace{3cm}\\ \hline
  7 &\hspace{5cm} & \hspace{3cm}\\ \hline
  8 &\hspace{5cm} & \hspace{3cm}\\ \hline
  9 &\hspace{5cm} & \hspace{3cm}\\ \hline 
  10 &\hspace{5cm} & \hspace{3cm}\\ \hline
  11 &\hspace{5cm} & \hspace{3cm}\\ \hline
  12 &\hspace{5cm} & \hspace{3cm}\\ \hline
  13 &\hspace{5cm} & \hspace{3cm}\\ \hline
  14 &\hspace{5cm} & \hspace{3cm}\\ \hline
  15 &\hspace{5cm} & \hspace{3cm}\\ \hline
  16 &\hspace{5cm} & \hspace{3cm}\\ \hline
 \end{tabular}

\end{center}

\pagebreak

\begin{center}
 {\Huge \textbf{Punktetabelle Physik}} \vspace{2cm} \\
 \LARGE
 \begin{tabular}{|r|l|l|}
  \hline
  Gruppe & Punkte & Platz  \\ \hline
  1 &\hspace{5cm} & \hspace{3cm}\\ \hline 
  2 &\hspace{5cm} & \hspace{3cm}\\ \hline
  3 &\hspace{5cm} & \hspace{3cm}\\ \hline
  4 &\hspace{5cm} & \hspace{3cm}\\ \hline
  5 &\hspace{5cm} & \hspace{3cm}\\ \hline
  6 &\hspace{5cm} & \hspace{3cm}\\ \hline
  7 &\hspace{5cm} & \hspace{3cm}\\ \hline
  8 &\hspace{5cm} & \hspace{3cm}\\ \hline
  9 &\hspace{5cm} & \hspace{3cm}\\ \hline 
  10 &\hspace{5cm} & \hspace{3cm}\\ \hline
  11 &\hspace{5cm} & \hspace{3cm}\\ \hline
  12 &\hspace{5cm} & \hspace{3cm}\\ \hline
  13 &\hspace{5cm} & \hspace{3cm}\\ \hline
  14 &\hspace{5cm} & \hspace{3cm}\\ \hline
  15 &\hspace{5cm} & \hspace{3cm}\\ \hline
  16 &\hspace{5cm} & \hspace{3cm}\\ \hline
 \end{tabular}

\end{center}
\pagebreak

\begin{center}
 {\Huge \textbf{Punktetabelle Info}} \vspace{2cm} \\
 \LARGE
 \begin{tabular}{|r|l|l|}
  \hline
  Gruppe & Punkte & Platz  \\ \hline
  1 &\hspace{5cm} & \hspace{3cm}\\ \hline 
  2 &\hspace{5cm} & \hspace{3cm}\\ \hline
  3 &\hspace{5cm} & \hspace{3cm}\\ \hline
  4 &\hspace{5cm} & \hspace{3cm}\\ \hline
  5 &\hspace{5cm} & \hspace{3cm}\\ \hline
  6 &\hspace{5cm} & \hspace{3cm}\\ \hline
  7 &\hspace{5cm} & \hspace{3cm}\\ \hline
  8 &\hspace{5cm} & \hspace{3cm}\\ \hline
  9 &\hspace{5cm} & \hspace{3cm}\\ \hline 
  10 &\hspace{5cm} & \hspace{3cm}\\ \hline
  11 &\hspace{5cm} & \hspace{3cm}\\ \hline
  12 &\hspace{5cm} & \hspace{3cm}\\ \hline
  13 &\hspace{5cm} & \hspace{3cm}\\ \hline
  14 &\hspace{5cm} & \hspace{3cm}\\ \hline
  15 &\hspace{5cm} & \hspace{3cm}\\ \hline
  16 &\hspace{5cm} & \hspace{3cm}\\ \hline
 \end{tabular}

\end{center}

\end{document}
